%!TEX root = ../main.tex
\chapter{Halbleiter}
Das am häufigsten verwendete Halbleiterelement ist Silizium. Silizium befindet sich in der vierten Hauptgruppe, und besitzt somit 4 Elektronen auf der äußersten, energetisch niedrigsten Schale. Sie werden Valenzelektronen genannt. Das Siliziumatom ist danach bestrebt acht Elektronen auf der äußersten Schale zu besitzen. In einem reinen Siliziumkristall geht jedes Valenzelektron eine Verbindung mit einem Valenzelektron eines benachbarten Siliziumatoms ein. Das Elektronenpaar umkreist die Kerne beider Atome, somit wird jedes Atom von vier Elektronenpaaren umkreist. Das Atom befindet sich in einem energetisch günstigen Zustand. Sämtliche Elektronen sind gebunden, es kann kein Strom fließen.  Dieser Zustand kann aber nur bei einer Umgebungstemperatur von 0 Kelvin erreicht werden. Werden Halbleiter erwärmt, Schwingen die Elektronen immer stärker um ihre Ruhelage. Durch diese Schwingung können Elektronenpaarbindungen aufbrechen. An der Aufgebrochen stelle fehlt nun ein Elektron, und hinterlässt ein Loch beziehungsweise Defektelektron. Dieses Defektelektron wird als positiver Ladungsträger betrachtet.  Das herausgebrochene Elektron kann sich nun bei angelegter Spannung durch den Halbleiter bewegen. Defektelektronen und freibewegliche Elektronen entstehen schon bei Raumtemperatur. Es sind dennoch zu wenige freie Ladungsträger um sie in der Technik als Leiter einzusetzen, und zu viele Ladungsträger um sie als Isolator zu verwenden. 


\section{Dotierte Halbleiter}
Beim Dotieren werden in den Halbleiter Fremdatome eingebaut. Die elektrische Leitfähigkeit nimmt erheblich durch die Fremdatome zu, da sie Ladungsträger freisetzen. Zum Dotieren von vierwertigen Halbleitern eignen sich drei- und fünfwertige Fremdatome.
Beim Dotieren mit einem dreiwertigen Fremdatom können nicht alle Paarbindungen im Kristallgitter erzeugt werden. Das Fremdatom übt Anziehungskräfte auf ein Siliziumelektron aus ohne eine Verbindungspartner zur Verfügung zu stellen. Bei der unvollständigen Bindung bleibt ein Loch, beziehungsweise Defektelektron übrig. Man Spricht bei der Dotierung von dreiwertigen Atomen von Akzeptoren. Der Akzeptor stellt nach dem Vervollständigen seiner Bindungspaare eine Ortsfeste Negative Ladung da, ein Akzeptor Ion, da es mehr Elektronen als Protonen besitzt. Das gesamt System bleibt jedoch durch den dazugehörigen positiven Ladungsträger elektrisch Neutral. Freie Elektronen, die aufgrund von Wärmebewegung entstanden sind, können, das durch Dotieren entstandene Loch besetzten, sie Verursachen wiederum selber ein Defektelektron an ihrer ursprünglichen Stelle. Bei angelegter Spannung ist die Elektronenbewegung in Richtung Pluspol gerichtet. Somit wandern die Löcher in Gegenrichtung zum negativen Pol \cite{Fischer2016}.  Beim Dotieren mit dreiwertigen Atomen wird die Konzentration von positiven Ladungsträgern, gegenüber den reinen Halbleiter erhöht. Daher spricht man von einem positiven Halbleiter, kurz p-Halbleiter \cite{Stiny2018}.

Beim Dotieren mit einem fünfwertigen Fremdatom, existiert für das fünfte Valenzelektron kein Bindungspartner in der vierwertigen Gitterstruktur. Das Elektron kann sich von seinem Atom lösen. Aus diesem Grund spricht man bei fünfwertigen Atomen von Donatoren. Durch die Abgabe ihres überschüssigen Elektrons stellen sie positive Ortsfeste Donatoratome da. Dennoch bleibt auch hier das gesamt System elektrisch Neutral. Bei dieser Dotierungsform ist die Konzentration an negativen Ladungsträgern höher als bei einem reinen Halbleiter, daher spricht man von einem negativen Halbleiter, kurz n-Halbleiter.


